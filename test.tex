\documentclass[paper=a4, % Seitenformat
         fontsize=10pt,  % Schriftgröße
         oneside,        % einseitig
         headsepline,    % Trennlinie für die Kopfzeile
         notitlepage     % keine extra Titelseite
]{scrartcl}              % KOMA-Script Article
%------------------------------------------------------------------------
\usepackage[<options>]{scrlayer-scrpage}
\usepackage[automark]{scrlayer-scrpage}  % Seiten-Stil für scrartcl
\usepackage[top=25mm]{geometry}  		% Oberer Rand 25mm Einrückung
\usepackage[utf8]{inputenc}              % Eingabekodierungen
\usepackage[T1]{fontenc}                 % Eingabekodierungen
\usepackage[english,ngerman]{babel}      % Mehrsprachenumgebung, Hauptsprache Deutsch
\usepackage{setspace}                    % Zeilenabstand
\usepackage{latexsym}                    % Latex-Symbole
\usepackage{amsfonts,amssymb,amstext}    % Mathematische Formeln
\usepackage{bbm}                         % bbm Schriftart
\usepackage{graphicx}                    % Abbildungen einbinden
\usepackage{listings}					%Programmcode einbingen
\usepackage{xcolor}						% Farben
\usepackage{changepage}


\pagestyle{scrheadings}					% \documentclass[paper=a4, % Seitenformat
fontsize=10pt,  % Schriftgröße
oneside,        % einseitig
headsepline,    % Trennlinie für die Kopfzeile
notitlepage     % keine extra Titelseite
]{scrartcl}              % KOMA-Script Article
%------------------------------------------------------------------------

\usepackage[automark]{scrlayer-scrpage}  % Seiten-Stil für scrartcl
\usepackage[top=25mm]{geometry}  		% Oberer Rand 25mm Einrückung
\usepackage[utf8]{inputenc}              % Eingabekodierungen
\usepackage[T1]{fontenc}                 % Eingabekodierungen
\usepackage[english,ngerman]{babel}      % Mehrsprachenumgebung, Hauptsprache Deutsch
\usepackage{setspace}                    % Zeilenabstand
\usepackage{latexsym}                    % Latex-Symbole
\usepackage{amsfonts,amssymb,amstext}    % Mathematische Formeln
\usepackage{bbm}                         % bbm Schriftart
\usepackage{graphicx}                    % Abbildungen einbinden
\usepackage{listings}					%Programmcode einbingen
\usepackage{xcolor}						% Farben
\usepackage{changepage}


\pagestyle{scrheadings}					% Kopfzeilen nach scr-Standard		

% Definition von Befehlen, um die Vorlage nützlicher und verständlicher zu machen
\newcommand{\codeind}[1]{\begin{adjustwidth}{3.5mm}{}#1\end{adjustwidth}}
\newcommand{\includecode}[1]{\codeind{\lstinputlisting[style=codestyle, language=C]{"src/#1"}}}
\newcommand{\includecodewithfilename}[1]{Datei: \texttt{#1}\vspace*{-1.5mm}\includecode{#1}}
\newcommand{\ownline}{\vspace{.7em}\hrule\vspace{.7em}} 
\newcommand{\aufgabe}[1]{\section*{Aufgabe #1}}

% Definition von Farben für die Codeblöcke
\definecolor{codegreen}{rgb}{0,0.6,0}
\definecolor{codegray}{rgb}{0.5,0.5,0.5}
\definecolor{codeblue}{rgb}{0.0, 0.0, 1.0}
\definecolor{bgcolour}{rgb}{0.97,0.97,0.97}
\definecolor{codered}{rgb}{0.7, 0.13, 0.13}

% Definition eines Designs für die Codeblöcke
\lstdefinestyle{codestyle}{
backgroundcolor=\color{bgcolour},
commentstyle=\color{codegreen},
keywordstyle=\color{codeblue},
numberstyle=\tiny\color{codegray},
stringstyle=\color{codered},
basicstyle=\ttfamily,
numberstyle=\ttfamily\tiny\color{codegray},
breakatwhitespace=false,         
breaklines=true,                 
captionpos=b,
extendedchars=true                    
keepspaces=true,                 
numbers=left,                    
numbersep=5pt,                  
showspaces=false,                
showstringspaces=false,
showtabs=false,                  
tabsize=4	
}

% Erlaubt es uns, in Codeblöcken Umlaute zu verwenden
\lstset{literate=%
{Ö}{{\"O}}1
{Ä}{{\"A}}1
{Ü}{{\"U}}1
{ß}{{\ss}}1
{ü}{{\"u}}1
{ä}{{\"a}}1
{ö}{{\"o}}1
}


\parindent0em

\begin{document}

% Kopf des Dokuments
\includegraphics[width=0.90\textwidth]{images/logo.png} \\
\textbf{Übung zur Vorlesung Informatik I} \hfill{WS 2024/25} \\  
Fakultät für Angewandte Informatik \\
Institut für Informatik \\
\textsc{Prof. Dr. J. Hähner, J. Linne, H. Cui, V. Gerling, N. Kemper} \\
\mbox{} \\
{\large Übungsgruppe X} % Nummer der Übungsgruppe einfügen
\ownline
\begin{center}
{\LARGE \textbf{Abgabe des X. Übungsblatts}} \\ % Nummer des Blatts einfügen
\mbox{} \\
{\large Namen der Teammitglieder} \\ % Namen der Teammitglieder einfügen
\end{center}
\ownline


%%%%%%%%%%%%%%%%%%%%%%%%%%%%%%%%%%%%%%%%%%%%%%%%%%%%%%%%%%%%%%%%%%%
%%%%%%%%%%%%%%%%%%% AB HIER BEARBEITEN %%%%%%%%%%%%%%%%%%%%%%%%%%%%
%%%%%%%%%%%%%%%%%%%%%%%%%%%%%%%%%%%%%%%%%%%%%%%%%%%%%%%%%%%%%%%%%%%

\aufgabe{X}

Im Folgenden wird die Syntax von LaTeX beschrieben.
Eine Leerzeile im Code erzeugt einen Zeilenumbruch im Dokument.

Zum Beginn einer Aufgabe kann der Befehl \textbackslash\texttt{aufgabe} genutzt werden.

Für \textbf{fette} und \textit{kursive} Schrift gibt es die Befehle \textbackslash\texttt{textbf} und \textbackslash\texttt{textit}.

Das \$-Zeichen markiert Beginn und Ende des Inline-Mathemodus.
Im Mathemodus lassen sich Formeln beschreiben:
Sei $\phi = \{ 2x \mid x \in \mathbbm{N}_0 \}$.

Eckige Klammern markieren Beginn und Ende des Display-Mathemodus.
In diesem werden z.B. Brüche größer dargestellt.
\[ \tan x = \frac{\sin x}{ \cos x} \]
Im Inline-Mathemodus: $\tan x = \frac{\sin x}{ \cos x}$
S1
\bigskip
Für Abstände bieten sich die Befehle \textbackslash\texttt{medskip} und \textbackslash\texttt{bigskip} an.

\bigskip
Grafiken lassen sich mit \textbackslash\texttt{includegraphics} einbinden.

\begin{center}S1S1
\includegraphics[width=0.3\textS1width]{images/abbildung.jpg}
\end{center}

Dabei werden verschiedene Formate, z.B. jpg, png oder pdf, unterstützt.
Mit Hilfe zusätzlicher Pakete können auch Grafiken in anderen Formaten, z.B. svg eingebunden werden.

\bigskip
Einen Seitenumbruch kann man mit \textbackslash\texttt{newpage} erzwingen.
\newpage


Diese \LaTeX-Vorlage bietet mit den Befehlen \textbackslash\texttt{includecode} und \textbackslash\texttt{includecodewithfilename} einfache Möglichkeiten Code aus einer Datei als ansprechend gestalteten Codeblock einzufügen. Die einzufügende Datei muss dazu im Ordner \texttt{src} gespeichert sein.
\includecode{programm.c}

\bigskip

Zum Beispiel so:
\aufgabe{1}
\begin{enumerate}
\item[a)]
\includecodewithfilename{programm.c}
\includecodewithfilename{programm2.c}
\item[b)] \dots
\end{enumerate}

%%%%%%%%%%%%%%%%%%%%%%%%%%%%%%%%%%%%%%%%%%%%%%%%%%%%%%%%%%%%%%%%%%%
%%%%%%%%%%%%%%%%%%%%%%%%%%%%%%%%%%%%%%%%%%%%%%%%%%%%%%%%%%%%%%%%%%%
%%%%%%%%%%%%%%%%%%%%%%%%%%%%%%%%%%%%%%%%%%%%%%%%%%%%%%%%%%%%%%%%%%%

\end{document}		Kopfzeilen nach scr-Standard		

% Definition von Befehlen, um die Vorlage nützlicher und verständlicher zu machen
\newcommand{\codeind}[1]{\begin{adjustwidth}{3.5mm}{}#1\end{adjustwidth}}
\newcommand{\includecode}[1]{\codeind{\lstinputlisting[style=codestyle, language=C]{"src/#1"}}}
\newcommand{\includecodewithfilename}[1]{Datei: \texttt{#1}\vspace*{-1.5mm}\includecode{#1}}
\newcommand{\ownline}{\vspace{.7em}\hrule\vspace{.7em}} 
\newcommand{\aufgabe}[1]{\section*{Aufgabe #1}}

% Definition von Farben für die Codeblöcke
\definecolor{codegreen}{rgb}{0,0.6,0}
\definecolor{codegray}{rgb}{0.5,0.5,0.5}
\definecolor{codeblue}{rgb}{0.0, 0.0, 1.0}
\definecolor{bgcolour}{rgb}{0.97,0.97,0.97}
\definecolor{codered}{rgb}{0.7, 0.13, 0.13}

% Definition eines Designs für die Codeblöcke
\lstdefinestyle{codestyle}{
	backgroundcolor=\color{bgcolour},
	commentstyle=\color{codegreen},
	keywordstyle=\color{codeblue},
	numberstyle=\tiny\color{codegray},
	stringstyle=\color{codered},
	basicstyle=\ttfamily,
	numberstyle=\ttfamily\tiny\color{codegray},
	breakatwhitespace=false,         
	breaklines=true,                 
	captionpos=b,
	extendedchars=true                    
	keepspaces=true,                 
	numbers=left,                    
	numbersep=5pt,                  
	showspaces=false,                
	showstringspaces=false,
	showtabs=false,                  
	tabsize=4	
}

% Erlaubt es uns, in Codeblöcken Umlaute zu verwenden
\lstset{literate=%
	{Ö}{{\"O}}1
	{Ä}{{\"A}}1
	{Ü}{{\"U}}1
	{ß}{{\ss}}1
	{ü}{{\"u}}1
	{ä}{{\"a}}1
	{ö}{{\"o}}1
}


\parindent0em

\begin{document}

% Kopf des Dokuments
\includegraphics[width=0.90\textwidth]{images/logo.png} \\
\textbf{Übung zur Vorlesung Informatik I} \hfill{WS 2024/25} \\  
Fakultät für Angewandte Informatik \\
Institut für Informatik \\
\textsc{Prof. Dr. J. Hähner, J. Linne, H. Cui, V. Gerling, N. Kemper} \\
\mbox{} \\
{\large Übungsgruppe X} % Nummer der Übungsgruppe einfügen
\ownline
\begin{center}
	{\LARGE \textbf{Abgabe des X. Übungsblatts}} \\ % Nummer des Blatts einfügen
	\mbox{} \\
	{\large Namen der Teammitglieder} \\ % Namen der Teammitglieder einfügen
\end{center}
\ownline


%%%%%%%%%%%%%%%%%%%%%%%%%%%%%%%%%%%%%%%%%%%%%%%%%%%%%%%%%%%%%%%%%%%
%%%%%%%%%%%%%%%%%%% AB HIER BEARBEITEN %%%%%%%%%%%%%%%%%%%%%%%%%%%%
%%%%%%%%%%%%%%%%%%%%%%%%%%%%%%%%%%%%%%%%%%%%%%%%%%%%%%%%%%%%%%%%%%%

\aufgabe{X}

Im Folgenden wird die Syntax von LaTeX beschrieben.
Eine Leerzeile im Code erzeugt einen Zeilenumbruch im Dokument.

Zum Beginn einer Aufgabe kann der Befehl \textbackslash\texttt{aufgabe} genutzt werden.

Für \textbf{fette} und \textit{kursive} Schrift gibt es die Befehle \textbackslash\texttt{textbf} und \textbackslash\texttt{textit}.

Das \$-Zeichen markiert Beginn und Ende des Inline-Mathemodus.
Im Mathemodus lassen sich Formeln beschreiben:
Sei $\phi = \{ 2x \mid x \in \mathbbm{N}_0 \}$.

Eckige Klammern markieren Beginn und Ende des Display-Mathemodus.
In diesem werden z.B. Brüche größer dargestellt.
\[ \tan x = \frac{\sin x}{ \cos x} \]
Im Inline-Mathemodus: $\tan x = \frac{\sin x}{ \cos x}$

\bigskip
Für Abstände bieten sich die Befehle \textbackslash\texttt{medskip} und \textbackslash\texttt{bigskip} an.

\bigskip
Grafiken lassen sich mit \textbackslash\texttt{includegraphics} einbinden.


Dabei werden verschiedene Formate, z.B. jpg, png oder pdf, unterstützt.
Mit Hilfe zusätzlicher Pakete können auch Grafiken in anderen Formaten, z.B. svg eingebunden werden.

\bigskip
Einen Seitenumbruch kann man mit \textbackslash\texttt{newpage} erzwingen.
\newpage


Diese \LaTeX-Vorlage bietet mit den Befehlen \textbackslash\texttt{includecode} und \textbackslash\texttt{includecodewithfilename} einfache Möglichkeiten Code aus einer Datei als ansprechend gestalteten Codeblock einzufügen. Die einzufügende Datei muss dazu im Ordner \texttt{src} gespeichert sein.
\includecode{programm.c}

\bigskip

Zum Beispiel so:
\aufgabe{1}
\begin{enumerate}
	\item[a)]
		\includecodewithfilename{programm.c}
		\includecodewithfilename{programm2.c}
	\item[b)] \dots
\end{enumerate}

%%%%%%%%%%%%%%%%%%%%%%%%%%%%%%%%%%%%%%%%%%%%%%%%%%%%%%%%%%%%%%%%%%%
%%%%%%%%%%%%%%%%%%%%%%%%%%%%%%%%%%%%%%%%%%%%%%%%%%%%%%%%%%%%%%%%%%%
%%%%%%%%%%%%%%%%%%%%%%%%%%%%%%%%%%%%%%%%%%%%%%%%%%%%%%%%%%%%%%%%%%%

\end{document}		